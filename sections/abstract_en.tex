%% LaTeX2e class for student theses
%% sections/abstract_en.tex
%% 
%% Karlsruhe Institute of Technology
%% Institute for Program Structures and Data Organization
%% Chair for Software Design and Quality (SDQ)
%%
%% Dr.-Ing. Erik Burger
%% burger@kit.edu
%%
%% Version 1.4, 2023-06-19

\Abstract
The aim of this study was twofold. Firstly, to identify performance issues encountered when implementing Confidential Computing methods, specifically Intel's Trusted Domain Extensions (TDX). Various inference benchmarks of well-known Large Language Models were executed on Trusted Domains and normal virtual machines to assess performance differences. Secondly, the study investigated how to establish a secure connection to a Trusted Domain. This involved explaining how TDX works, determining the feasibility of a secure connection and Trusted Domain identification, followed by attempts to practically implement these theoretical findings with a Trusted Domain in the Azure Cloud. Results demonstrated that the performance losses due to Intel TDX were relatively small compared to the overall losses caused by additional layers due to the usage of virtual machines. Furthermore, it was observed that not all security assurances given by Intel, especially that the cloud provider no longer needs to be trusted, are guaranteed, even with perfect implementation of all TDX specifications. In practice, it was additionally found that even the limited security assurances still depend on cooperation from the cloud provider. While it is recognizable if the cloud provider does not adhere to all of Intel's requirements, their collaboration remains essential.