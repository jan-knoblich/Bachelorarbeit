%% LaTeX2e class for student theses
%% sections/content.tex
%% 
%% Karlsruhe Institute of Technology
%% Institute for Program Structures and Data Organization
%% Chair for Software Design and Quality (SDQ)
%%
%% Dr.-Ing. Erik Burger
%% burger@kit.edu
%%
%% Version 1.4, 2023-06-19

\chapter{Introduction}
\label{ch:Introduction}

\begin{fancyquotes}
Privacy is not something that I’m merely entitled to, it’s an absolute prerequisite. 

\textit{Marlon Brando}
\end{fancyquotes}

Privacy becomes increasingly more prevalent in times, where we put more and more information about ourselves on the Internet. While many people give no regard to the amount of data they put onto the internet for free, this thesis will focus less on what people are willingly sharing but how to protect data that we want to protect. Data can usually be in one of three states requiring protection. Proven procedures for the secure transmission and storage of data have existed for some time. In the past, however, it was a challenge to fully protect data from unauthorized access during processing. This situation has changed fundamentally over the last ten years. Confidential computing refers to concepts and technologies that are designed to protect data and applications, even if they are executed on third-party hardware. This thesis focuses on Intel's Trust Domain Extensions (TDX) from 2023, an extension of the Software Guard Extension (SGX) from 2015. These offer developers the option of using hardware-based encryption. In the case of SGX, selected memory areas and application code are isolated; TDX extends this to include hardware-based isolation of the entire virtual machine (VM) from the hypervisor, among other things. 
A fundamental problem in the security industry is the trade-off between performance and security. For this reason, this bachelor thesis is dedicated to analyzing the impact of using these technologies on the performance of a Large Language Model (LLM). This thesis contributes to this goal by answering the following questions:
\begin{itemize}
    \item What is the impact of TDX on an applications performance? How do they compare to general performance decreases of using an application in a VM?
\item What security assumptions need to be made for a system using Intel TDX to be secure? What does it take to securely connect to a TD and what risks still remain?
\end{itemize}
To answer these two questions, the TDX specifications and attestation flow was investigated and explained. The limits of the threat model that is in use by Intel and the Confidential Computing Consortium were outlined, this was followed by looking at a practical implementation of a TD, its quote generation and general attestation flow. This data was then used to differentiate between the theoretical and practical dangers of TDX.
To answer the performance questions, numerous benchmarks were run using Python and Huggingface transformers.
