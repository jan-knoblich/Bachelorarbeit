%% LaTeX2e class for student theses
%% sections/abstract_de.tex
%% 
%% Karlsruhe Institute of Technology
%% Institute for Program Structures and Data Organization
%% Chair for Software Design and Quality (SDQ)
%%
%% Dr.-Ing. Erik Burger
%% burger@kit.edu
%%
%% Version 1.4, 2023-06-19

\Abstract
Das Ziel dieser Arbeit war zweischneidig. Zum Einen galt es herauszufinden, welche Performanceprobleme bei dem Einsatz von Confidential Computing Methoden, spezifisch Intels Trusted Domain Extensions, auftreten. Hierfür wurden verschiedene Inference Benchmarks bekannter Large Language Models auf Trusted Domains und normalen virtuellen Maschinen ausgeführt. Zum Anderen wurde untersucht wie es möglich ist eine sichere Verbindung zu einer Trusted Domain aufzubauen. Hierfür musste zunächst erklärt werden, wie TDX überhaupt funktioniert, ob eine sichere Verbindung, inklusive Identifizierung einer Trusted Domain überhaupt möglich ist und danach wurde versucht diese theoretischen Erkenntnisse mit einer TD der Azure Cloud in die Tat umzusetzen. Es konnte gezeigt werden, dass die performance Einbußen durch Intel TDX im Vergleich zu den Gesamteinbußen durch die weitere Schicht, die eine Virtuelle Maschine immer mit sich bringt relativ klein sind. Darüber hinaus konnte festgestellt werden, dass nicht alle Sicherheitsversprechen die Intel gibt, vor allem, dass dem Cloud Provider nicht mehr vertraut werden muss, selbst bei perfekter Umsetzung aller TDX Spezifikationen nicht gegeben ist. In der Praxis zeigte sich dann zusätzlich, dass selbst die eingeschränkten Sicherheitsversprechen immer noch auf die Kooperation durch den Cloud Provider angewiesen sind. Es ist zwar nachvollziehbar, dass der Cloud Provider nicht alles so umsetzt, wie von Intel gefordert aber seine Mitarbeit ist weiterhin von Nöten.