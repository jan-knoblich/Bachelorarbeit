%% LaTeX2e class for student theses
%% thesis.tex
%% 
%% Karlsruhe Institute of Technology
%% Institute for Program Structures and Data Organization
%% Chair for Software Design and Quality (SDQ)
%%
%% Dr.-Ing. Erik Burger
%% burger@kit.edu
%%
%% See https://sdq.kastel.kit.edu/wiki/Dokumentvorlagen
%%
%% Version 1.4, 2023-06-19

%% Available page modes: oneside, twoside
%% Available languages: english, ngerman
%% Available modes: draft, final (see README)
\documentclass[twoside, english]{sdqthesis}

%% ---------------------------------
%% | Information about the thesis  |
%% ---------------------------------

%% Name of the author
\author{Jan Knoblich}

%% Title (and possibly subtitle) of the thesis
\title{Evaluation of Intel Confidential Computing Technologies with Regard to their Performance, Security and Resulting Application Possibilities}

%% Type of the thesis 
\thesistype{Bachelor's Thesis}

%% Change the institute here, ``KASTEL'' is default
\myinstitute{KASTEL – Institute of Information Security and Dependability
Cryptography and Security Group}

%% You can put a logo in the ``logos'' directory and include it here
%% instead of the SDQ logo
% \grouplogo{myfile}
%% Alternatively, you can disable the group logo
% \nogrouplogo

%% The reviewers are the professors that grade your thesis
\reviewerone{Prof. Dr. Jörn Müller-Quade}
\reviewertwo{Prof. Dr. Thorsten Strufe}

%% The advisors are PhDs or Postdocs
\advisorone{Jeremias Mechler, M.Sc. }
%% The second advisor can be omitted
\advisortwo{Felix Dörre, M.Sc.}

%% Please enter the start end end time of your thesis
\editingtime{17. October 2023}{12. March 2024}

\settitle

%% --------------------------------
%% | Bibliography                 |
%% --------------------------------

%% Use biber instead of BibTeX, see README
\usepackage[citestyle=numeric,style=numeric,backend=biber]{biblatex}
\addbibresource{thesis.bib}

%% For example texts -- please remove in the final version
\usepackage[acronym]{glossaries}
\usepackage{todonotes}
\usepackage{listings}

\usepackage[T1]{fontenc}
\usepackage{array}

\colorlet{punct}{red!60!black}
\definecolor{background}{HTML}{EEEEEE}
\definecolor{delim}{RGB}{20,105,176}
\definecolor{black}{RGB}{0,0,0}
\colorlet{numb}{magenta!60!black}

\thispagestyle{empty}
\usepackage{lipsum}
\usepackage{tikz}
\usetikzlibrary{backgrounds}
\makeatletter

\tikzset{%
  fancy quotes/.style={
    text width=\fq@width pt,
    align=justify,
    inner sep=1em,
    anchor=north west,
    minimum width=\linewidth,
  },
  fancy quotes width/.initial={.8\linewidth},
  fancy quotes marks/.style={
    scale=8,
    text=white,
    inner sep=0pt,
  },
  fancy quotes opening/.style={
    fancy quotes marks,
  },
  fancy quotes closing/.style={
    fancy quotes marks,
  },
  fancy quotes background/.style={
    show background rectangle,
    inner frame xsep=0pt,
    background rectangle/.style={
      fill=gray!25,
      rounded corners,
    },
  }
}

\newenvironment{fancyquotes}[1][]{%
\noindent
\tikzpicture[fancy quotes background]
\node[fancy quotes opening,anchor=north west] (fq@ul) at (0,0) {``};
\tikz@scan@one@point\pgfutil@firstofone(fq@ul.east)
\pgfmathsetmacro{\fq@width}{\linewidth - 2*\pgf@x}
\node[fancy quotes,#1] (fq@txt) at (fq@ul.north west) \bgroup}
{\egroup;
\node[overlay,fancy quotes closing,anchor=east] at (fq@txt.south east) {''};
\endtikzpicture}

\makeatother


\lstdefinelanguage{jsonmain}{
    basicstyle=\small\ttfamily,
    numbers=left,
    numberstyle=\scriptsize,
    stepnumber=1,
    numbersep=8pt,
    showstringspaces=false,
    breaklines=true,
    frame=lines,
    backgroundcolor=\color{background},
    literate=
     *{0}{{{\color{black}0}}}{1}
      {1}{{{\color{black}1}}}{1}
      {2}{{{\color{black}2}}}{1}
      {3}{{{\color{black}3}}}{1}
      {4}{{{\color{black}4}}}{1}
      {5}{{{\color{black}5}}}{1}
      {6}{{{\color{black}6}}}{1}
      {7}{{{\color{black}7}}}{1}
      {8}{{{\color{black}8}}}{1}
      {9}{{{\color{black}9}}}{1}
      {:}{{{\color{punct}{:}}}}{1}
      {,}{{{\color{punct}{,}}}}{1}
      {\{}{{{\color{delim}{\{}}}}{1}
      {\}}{{{\color{delim}{\}}}}}{1}
      {[}{{{\color{delim}{[}}}}{1}
      {]}{{{\color{delim}{]}}}}{1},
}
\lstdefinelanguage{json}{%
  language     = jsonmain,
  basicstyle=\footnotesize\ttfamily,
  numbers=none,
}
\usepackage{cleveref}
\usepackage[newfloat]{minted}


\newacronym{tcb}{TCB}{Trusted Computing Base}
\newacronym{td}{TD}{Trusted Domain}
\newacronym{tdx}{TDX}{Trusted Domain Extensions}
\newacronym{NP-SEAMLDR}{NP-SEAMLDR}{\Gls{NP-SEAMLDR1}}
\newacronym{P-SEAMLDR}{P-SEAMLDR}{\Gls{P-SEAMLDR1}}
\newacronym{OS}{OS}{Operating System}
\makeglossaries


%% ====================================
%% ====================================
%% ||                                ||
%% || Beginning of the main document ||
%% ||                                ||
%% ====================================
%% ====================================
\begin{document}


%% Set PDF metadata
\setpdf

%% Set the title
\maketitle

%% The Preamble begins here
\frontmatter

\input{sections/declaration.tex}

\setcounter{page}{1}
\pagenumbering{roman}

%% ----------------
%% |   Abstract   |
%% ----------------
 
%% For theses written in English, an abstract both in English
%% and German is mandatory.
%%
%% For theses written in German, a German abstract is sufficient.
%%
%% The text is included from the following files:
%% - sections/abstract

\includeabstract


%% ------------------------
%% |   Table of Contents  |
%% ------------------------
\tableofcontents


\printglossaries

\listoffigures
\listoftables

%% -----------------
%% |   Main part   |
%% -----------------

\mainmatter
\newcommand{\myparagraph}[1]{\paragraph*{#1}\mbox{}\\}
\newcommand{\guillemet}[1]{\guillemotleft #1 \guillemotright}
%% LaTeX2e class for student theses
%% sections/content.tex
%% 
%% Karlsruhe Institute of Technology
%% Institute for Program Structures and Data Organization
%% Chair for Software Design and Quality (SDQ)
%%
%% Dr.-Ing. Erik Burger
%% burger@kit.edu
%%
%% Version 1.4, 2023-06-19

\chapter{Introduction}
\label{ch:Introduction}

\begin{fancyquotes}
Privacy is not something that I’m merely entitled to, it’s an absolute prerequisite. 

\textit{Marlon Brando}
\end{fancyquotes}

Privacy becomes increasingly more prevalent in times, where we put more and more information about ourselves on the Internet. While many people give no regard to the amount of data they put onto the internet for free, this thesis will focus less on what people are willingly sharing but how to protect data that we want to protect. Data can usually be in one of three states requiring protection. Proven procedures for the secure transmission and storage of data have existed for some time. In the past, however, it was a challenge to fully protect data from unauthorized access during processing. This situation has changed fundamentally over the last ten years. Confidential computing refers to concepts and technologies that are designed to protect data and applications, even if they are executed on third-party hardware. This thesis focuses on Intel's Trust Domain Extensions (TDX) from 2023, an extension of the Software Guard Extension (SGX) from 2015. These offer developers the option of using hardware-based encryption. In the case of SGX, selected memory areas and application code are isolated; TDX extends this to include hardware-based isolation of the entire virtual machine (VM) from the hypervisor, among other things. 
A fundamental problem in the security industry is the trade-off between performance and security. For this reason, this bachelor thesis is dedicated to analyzing the impact of using these technologies on the performance of a Large Language Model (LLM). This thesis contributes to this goal by answering the following questions:
\begin{itemize}
    \item What is the impact of TDX on an applications performance? How do they compare to general performance decreases of using an application in a VM?
\item What security assumptions need to be made for a system using Intel TDX to be secure? What does it take to securely connect to a TD and what risks still remain?
\end{itemize}
To answer these two questions, the TDX specifications and attestation flow was investigated and explained. The limits of the threat model that is in use by Intel and the Confidential Computing Consortium were outlined, this was followed by looking at a practical implementation of a TD, its quote generation and general attestation flow. This data was then used to differentiate between the theoretical and practical dangers of TDX.
To answer the performance questions, numerous benchmarks were run using Python and Huggingface transformers.

\chapter{Preliminaries}
\label{ch:Foundations}

\section{Confidential Computing}
\label{sec:Foundations:ConfComputing}
Confidential Computing means the protection of data during computation by performing those computations in attested Trusted Execution Environments (TEE). Similarly to Trusted Computing, Confidential Computing is predominantly a marketing term and has no single definition. The following tries to establish the definition as it will be used in this thesis. Unlike Trusted Computing, which aims to establish integrity of hardware and software, Confidential Computing also aims to guarantee the authenticity. Using TEEs in the lowest layer of hardware reduces the number of trusted parties. In this case, a party could be a hardware or software vendor. Security in one abstraction layer of the compute stack is only as strong as the layers below it if no other measures are used to prevent access. Homomorphic encryption gives some additional security guarantees even if the underlying layers are not trusted. Figure \ref{fig:computestack} shows a basic compute stack with the most common layers. A bug in the hypervisor, if present, can affect the execution of the operating system and, in the worst case allow unauthorized entities to read or change the data in use.
\begin{figure}
\centering
\includegraphics[width=0.5\textwidth]{figures/ComputeStack.png}
\caption{A rudimentary Compute Stack. Most layers can be split up further if needed to}
\label{fig:computestack}
\end{figure}Security in higher layers can be bypassed from lower layers. Security solutions at the hardware level can remove the operating system, device drivers, and more from the list of trusted parties. A TEE is, as defined by the Confidential Computing Consortium, an environment that provides assurance in three different properties:
\begin{itemize} 
    \item \guillemotright Data confidentiality: Unauthorized entities cannot view data while it is in use within the TEE
    \item Data integrity: Unauthorized entities cannot alter the data while it is in use within the TEE.
    \item Code integrity: Unauthorized entities cannot alter the code executing in the TEE.\guillemotleft \cite{noauthor_ccc_outreach_whitepaper_updated_november_2022pdf_2023}
\end{itemize} 
In the context of Confidential Computing, unauthorized entities include, but are not limited to, other applications on the host, the host operating system itself, as well as the hypervisor, the infrastructure owner, and anyone else with physical hardware access. 
Combined, these three attributes provide an assurance that the data are kept confidential and the calculations that are performed are correct.
Depending on the specific TEE used, there might be additional security properties, but the TEEs covered in this thesis do not have more\cite{noauthor_ccc_outreach_whitepaper_updated_november_2022pdf_2023}.

\myparagraph{Integrity Measurement Architecture}

\label{IMA}

Integrity Measurement Architecture (IMA) is a feature of the Linux kernel that ensures system integrity while the system is running. It computes the hashes of files and programs prior to their execution, provides reporting capabilities, and verifies that they adhere to a predefined list.\cite{Luo_container_ima}. IMA is recommended by the Trusted Computing Group for identity verification \cite{trusted_computing_group_tcg_2023}. According to the IMA wiki \url{https://sourceforge.net/p/linux-ima/wiki/Home/} it contains two subsystems: Measurement and Appraisal. Measurement is responsible for gathering, storing, and verifying measurements. Appraisal conducts the evaluation and auditing tasks by comparing a collected hash with a stored hash and prevents access if they do not match. IMA depends on Trusted Platform Module (TPM)\cite{Luo_container_ima}. It needs a custom initramfs or similar early boot process script capabilities to be used.

\section{Trusted Domain Extension}
One implementation of TEEs by Intel is called Trusted Domain Extensions (TDX). This section gives an overview of its building blocks, its system architecture, memory protection mechanisms, I/O model, attestation, and future features.
\subsection{TDX Building Blocks}
\label{sec:tdxBuildingBlocks}
TDX combines and extends several existing Intel technologies, including Virtualization Technology (VT), Total Memory Encryption (TME), and Software Guard Extensions (SGX). This section will give an overview the technologies and how they are used. The building blocks were also outlined in \cite{cheng_intel_2023}, to which this will give some more context.
\subsubsection{Software Guard Extension}
SGX is a set of instruction codes, which were Intels first trusted execution environment. It was intended to protect against memory bus snooping and cold boot attacks. It is still in use today to protect code and data within a so called enclave, which is a secure container, containing sensitive code and data. It can contain entire applications or even an operating system but was intended to house just small parts of an application. It is encrypted using Memory Protection Extensions a predecessor to Total Memory Encryption, which is explained in \cref{Memory Encrpytion}. Data is only decrypted inside the processor\cite{intel_corporation_overview--intel-sgx-enclave_nodate}. SGX aims to protect the confidentiality of the computation inside an enclave. SGX uses hardware-based memory encryption to ensure that the RAM of an enclave can only be accessed by authorized code. Any unauthorized attempt to access the memory will result in an exception. SGX offers both local and, more importantly, remote attestation to verify the integrity of enclaves. Local attestation establishes trust between two local enclaves, and remote attestation verifies trustworthiness to a third party entity. More on attestation can be found in \cref{TDX attestation}. On the same platform, TDX and SGX are within the same Trusted Computing Base (TCB).  Therefore, they can attest each other locally\cite{intel_corporation_dcap_2024-1}, which will become important for TDX in \cref{Pre-Attestation setup}. During its lifetime Intel SGX has had many security issues, with most of them being preventable by the application developer themself. Van Schaik et Al. provide an overview of known issues at the time of writing and ways to mitigate them \cite{sgxfail}. The same source also contains an overview of attacks that can only be prevented with microcode updates by Intel itself. SGX does not want to protect against hardware-based attacks in general but mitigates some basic ones \cite{costan_intel_2016}. What is considered a basic hardware attack will be discussed further in \cref{Threat_Model}. It wants to protect against unpriviliged and privileged software attacks as well as startup code attacks \cite{schutz_general_nodate}.
\subsubsection{Intel Virtualization Technology}
Intel VT is a set of hardware-assisted virtualization features in Intel processors that can provide improved performance, isolation, and security compared to software-based virtualization. Intel VTs features include virtualization of the CPU, memory and I/O.
Intel processors contain, depending on the architecture, either VT-x or VT-i instructions. The latter being only used on the discontinued Itanium architecture. Processors with VT-x feature an extra instruction set known as Virtual Machine Extensions (VMX), allowing virtualization control. Virtualization allows the usage of multiple isolated Virtual Machines on the same hardware. It can also allow multiple different operating systems to run in these VMs\cite{intel_corporation_intel_nodate}. VMX has two different execution modes: VMX root mode, used by the hypervisor, and VMX nonroot mode, used by the guest VMs. VMX uses a second-level address translation called Extended Page Table (EPT), which aims to eliminate the overhead from software-managed page tables \cite{uhlig_intel_2005}. The TDX Module (TDXM) runs in the new Secure Arbitration Mode (SEAM) VMX root and the TDs run in the nonroot mode. The biggest change from normal VMX to SEAM VMX is the usage of an additional Extended Page Table. With VMX the hypervisor holds only one table per guest kernel, the TDXM has two per guest TD. A protected one for private memory and another unprotected one for shared memory.
With TDX being a virtual machine-based TEE, it depends on VT to ensure isolation between its Trusted Domains (TDs)\cite{cheng_intel_2023}.
\subsubsection{Intel Total Memory Encryption}
\label{Memory Encrpytion}
Intel Total Memory Encryption (TME) is a security measure designed to protect against attackers who have physical access to the memory of a computer or direct access via a host system and attempt to steal data. TME aims to protect against some, for example bus memory probing, but not all physical attacks. Direct Memory Access attacks for example are not included.TME encrypts the entire computer's memory using a single key, which is generated at boot-time by a hardened hardware-based random number generator. Memory encryption is performed by encryption engines on each memory controller, using the recommended standard AES-XTS, for storage encryption, of the US National Institute of Standards and Technologies \cite{morris_dworkin_recommendation_2015} and the German Federal Office for Information Security with 128 or 256-bit keys\cite[~p. 24]{bundesamt_fur_sicherheit_in_der_informationstechnik_cryptographic_2023}. The encryption engines sit directly in between the memory controllers and the CPU cache, meaning the data inside the SoC remains plain text, while the data inside the memory is always encrypted. Total Memory Encryption Multi Key (TME-MK, also sometimes MKTME) extends TME to support multiple keys and memory encryption at page granularity. To use TME-MK in virtualized environments, the hypervisor must be trusted, which violates the threat model for confidential computing, which will be explained in more detail in \cref{Threat_Model}. Therefore, in TDX, the TDX Module is responsible for controlling the memory encryption of TDs. Figure \ref{fig:component-overview} shows what a TD encapsulates. The TDX Module requests the processor to generate a new key when building a new TD and binds the two together via its Key Management Tables. The TDX module then utilizes MKTME to encrypt the cache lines. The TDXM stores cryptographic keys only in its Key Encryption Tables, thus never exposing them to the outside\cite{cheng_intel_2023}. 

\subsection{TDX System Architecture}
\label{TDX Architecture}
Fig \ref{fig:component-overview} illustrates the runtime architecture of TDX. It is made up of two key components: 
\begin{figure}
\centering
\includegraphics[width=\textwidth]{figures/TDX-Component-Overview}
\caption{TDX Component Overview taken from \cite[p.~19]{noauthor_tdx-module-10-public-specpdf_nodate}. Shown here is an extension of the Compute Stack in Fig. \ref{fig:computestack} with the colors corresponding to the non-TDX variants.}
\label{fig:component-overview}
\end{figure}
\begin{itemize}
    \item TDX-enabled processors, which offer combinations of the aforementioned capabilities.
    \item TDX Module, an Intel-signed and CPU-attested software module that leverages the features of TDX-enabled processors to facilitate the construction, execution, and termination of TDs while enforcing the security guarantees. These will be further looked at in the Threat Model in \cref{Threat_Model} and the Security Analaysis in \cref{Security Analysis}
\end{itemize}
Additonally it also shows the surrounding environment containing optionally normal VMs, the hardware and TDs ontop of the TDXM.The TDX Module provides two sets of interface functions, host-side interface functions for a TDX-enlightened hypervisor and guest-side interface functions for TDs. It is loaded and executed in the SEAM RANGE, which is a portion of system memory reserved via UEFI/BIOS. The P-SEAM Loader can install and update the TDX Module. More information on the loading process can be found in \cite{noauthor_white_nodate} and \cite{noauthor_tdx-module-10-public-specpdf_nodate}. The P-SEAMLDR will be discussed further in \ref{Security Analysis}.
The TDX-aware hypervisor operates in the conventional VMX root mode and uses the SEAMCALL instruction to invoke functions on the host side interface (function names start with TDH) of the TDX Module. Upon execution of the SEAMCALL instruction, the logical processor transitions from the VMX root mode to the SEAM VMX root mode and starts executing code within the TDX Module. Once the TDX module has finished its task, it returns to the hypervisor in VMX root mode by executing the SEAMRET instruction. On the other hand, TDs run in the SEAM VMX nonroot mode. TDs can trap into the TDX Module either through a TD exit by invoking the TDCALL instruction or triggered by some external event, e.g. an external interrupt or exception. In both cases, the logical processor transitions from the SEAM VMX nonroot mode into the SEAM VMX root mode and starts executing inside the TDX Module. \cref{fig:seamFigure} shows these transitions and their counter parts.

\begin{figure}
\centering
\includegraphics[width=0.7\textwidth]{figures/SEAMVMXÜBergänge.png}
\caption{VMX and SEAM VMX transitions. The colors are the same as they are for the implementations of these in \cref{fig:component-overview}. Note that it is not possible to transition from VMX non-root to SEAM non-root directly, as this would correspond to a direct transition from a TD to a non-TD vm.}
\label{fig:seamFigure}
\end{figure}
\subsection{TDX Key Generation and TD encryption}
The Host Hypervisor or Virtual Machine Manager (host VMM), which explicitly is not part of the TDX module but is aware of it calls the TDH.SYS.KEY.CONFIG function during startup, which configures the TDX modules global private key on the current CPU package \cite{intel_corporation_intel-tdx-module-15-abi-spec-348551001pdf_2024}. The host VMM can create a new guest TD by allocating and initializing a TD root control structure. The host assigns the TD a Host Key ID (HKID), which can be used to tag the memory accesses made by the TD\cite{noauthor_tdx-module-10-public-specpdf_nodate}. Subsequently, the host VMM calls the MKTME hardware, which encrypts the specified memory using a hardware-generated encryption key\cite{noauthor_multi-key-total-memory-encryption-spec-14pdf_nodate}. The VMM of the TD host configures the MKTME hardware by calling the TDH.MNG.KEY.CONFIG function on each CPU package. This will program the encryption key into the MKTME encryption engines\cite{noauthor_tdx-module-10-public-specpdf_nodate}. At this point, the TD private memory section is created and accessible by the TD. The VMM can then use Intel TDXM interface functions to create control structures. 

\subsection{The Threat Model}

\label{Threat_Model}

This section will outline the threat model, that is used in this thesis, as well as what Intel, in person of Andi Kleen and Elena Reshetova, outlined at \cite{elena_reshetova_intel_2023} sees as the TDX threat model.

\subsubsection{Adversarial Goals}

An attacker targeting TDX or a TD directly can focus on different components, depending on what their goals are. In general, they would be interested in leaking sensitive data or disrupting the expected behavior of a TD. These were also outlined by Aktas et Al. \cite{aktas_intel_nodate}.

\myparagraph{Leaking TD Secrets}

Once the TD has been verified via attestation, the owner of the TD usually provides confidential information to the TD. This could take the form of cryptographic keys, private user data, or similar. TDX aims to isolate this information from any malicious third party.

\myparagraph{Manipulating TD Behavior}

An adversary may be looking to change the behavior of a victim TD in various ways. As previously discussed the VMM can change some memory mappings the TD uses, which could lead to changes in the behavior of the application and faulty computational results.

\myparagraph{Host Denial-of-Service}

An adversary may aim to limit the availability of a cloud provider by obstructing the scheduling of other workloads (TDs, VMs) on a machine. This can be achieved by abusing the scheduler. Additional attacks can vary in their severity. For instance, a bug in a TD may lead to its shutdown, while a bug in the TDX module could necessitate an immediate halt of all TDs on the machine. Some memory states can cause unrecoverable machine checks which necessitates a full power cycle to recover \cite{aktas_intel_nodate}.

\subsubsection{Attack Vectors and Vulnerabilities}
In general attack categories, sometimes also called threat vectors, in scope for confidential computing according to the CCC are as follows:

\begin{itemize}
    \item \textbf{Software Attacks} aimed at firmware and software on the host and also software inside the TEE
    \item \textbf{Protocol Attacks} on the attestation protocols and data in transport
    \item \textbf{Cryptogrphic Attacks} against ciphers and algorithms, for example hashing or encryption algorithms
    \item \textbf{Basic physical attacks} such as \guillemotright cold DRAM extraction, bus and cache monitoring and plugging of attack devices into an existing port \guillemotleft \cite{confidential_computing_consortium_ccc--technical-analysis--confidential-computing-v13_unlockedpdf_2023}
\end{itemize}

The threat model does not address any threats made possible by the TDX guest userspace directly using TDVMCALL hypercalls (through the TDX-module) and shared memory for IO exposed to an untrusted host/VMM. If TDX guest userspace enables debug or test tools that can read memory-mapped I/O (MMIO) or PCI configuration space without validating the input from untrusted host/VMM, it opens up the possibility of many more attacks. In addition malicious input from, for example,  PortIO, SharedMemory or KVM Hypercalls is consumed from the host by device drivers in the TD, which are implicitly trusted and part of the TCB. If those drivers are vulnerable this can lead to threats from inside the TD, although Intel has provided mitigation measures against those \cite{elena_reshetova_intel_2023}.

According to Intel the TCB contains only a few components:

\begin{itemize}
    \item Intel TDX module, with Persistent Seam Loader (P-SEAMLDR) an Non-Persistent Seam Loader (NP-SEAMLDR)
    \item Intel Authenticated Code Modules (ACM), for example the BIOS ACM
    \item TD Quoting Enclave
    \item Intel CPU Hardware
\end{itemize}

All other components, such as the BIOS, SMM, host OS, and the VMM, are situated outside the TCB \cite{noauthor_tdx-whitepaper-february2022pdf_nodate}.


The TCB for Intel TDX has a restricted number of system components, which creates only a few opportunities for adversaries to breach system security. Moreover, when a malicious Virtual Machine Manager (VMM) and a malicious TD work together, they can exploit complex scenarios that the system designers may not have foreseen. The following will give a quick summary of the attack vectors from \cite{aktas_intel_nodate} and then describe two additional vectors, which are more prevalent for the issues at hand.

\myparagraph{Intel signed Software}

Even though the TCB is small, Intel has to build a chain of trust from its own hardware and the BIOS to the TDX Module and the guest TDs. It does that by using different open and closed source layers of signed code the system has to go through before it is fully booted. With the BIOS being outside the TCB, TDX has to make sure that all security-sensitive settings are within an acceptable range. Intel has developed MCHECK for this. MCHECK is situated inside the CPU microcode and is not made public by Intel, it is thus completely opaque to the user. Next, during the Host startup NP-SEAMLDR and P-SEAMLDR are loaded by trapping into the SEAM root-mode mentioned in \cref{TDX Architecture}. NP-SEAMLDR prevents the BIOS from executing any further. NP-SEAMLDR then loads P-SEAMLDR which in turn loads the TDX Module, which is the central privileged software component for runnings TDs. The security of these has been looked at and verified by Aktas et Al. prior to TDX beining released \cite{aktas_intel_nodate}. They have discovered at least ten different, software-based, security issues, which have been fixed now. One of these attacks allowed arbitrary code execution inside elevated SEAM root-mode, which would completely cancel any security guarantees by Intel. A malicious TD can only attack via the TDX Module, which if compromised would grant complete access to the TD. As of the writing of this thesis Intel has not published any security advisories regarding TDX \cite{security_advisories}.

\myparagraph{Hardware attacks}

Intel considers \guillemotright opening the case and tampering of [sic!] internal hardware to compromise SGX is out of scope for SGX threat model. \guillemotleft \cite{chen_voltpillager_nodate}. As outlined in \cref{sec:tdxBuildingBlocks} TDX relies on SGX for its security, thus vulnerabilities in SGX can also affect TDX. The threat model for Confidential Computing contains basic hardware attacks, with the difference between basic and complex hardware attacks being unclear. The amount of preparation and additional hardware needed appear to be a differentiating factor, some basic hardware attacks can be prevented via software precautions but both are only completely prevented by preventing direct hardware access. 

\myparagraph{Outside connection to the TD}

The last major interface is the outside connection to the TD. Here two different problems arise. Making sure that data is not compromised in transit is technically not part of the threats TDX wants to protect against and it does not itself do so,  but it can help with establishing a secure connection. The second issue that comes up is making sure that communication is done with the correct TD. TD identity checking is explicitly outside the scope of TDX but it can help with that and this will be another concern in this thesis. The TDX attestation only guarantees that somewhere a TD runs on certified hardware with a signed TDX Module version. Additional information the Quote can supply will be needed to identify a TD. Without a secure connection a TD remains useless, as any results it calculates, can not be consumed by an outsider. An exemplary man-in-the-middle attack is shown in \ref{fig:man_in_the_middle}.
\begin{figure}
   \centering
       \includegraphics[width=.75\textwidth]{figures/Man-In-The-Middle.png} 
 \caption{A simple man-in-the-middle attack against a TD. The attacker reroutes the challenge to an unsecured VM, which in turn requests a quote from a different TD}
 \label{fig:man_in_the_middle}
\end{figure}
Provided that security promises within the previously mentioned parts of the TCB are solid, connecting to and verifying the identity of a \Gls{TD} is the last step towards secure computation with user Input.

\subsubsection{Threats TDX does not protect against}

Some threats and attacks that are explicitly outside the threat model of TDX. For some of these TDX might hinder the effectiveness or even prevent them but it does not try do that.

\myparagraph{Side-Channel Attacks}

Side-Channel Attacks are attacks that take advantage of unintended output channels of a system. The first documented one was when in 1965 MI5 tried to decipher a cryptography device used in the Egyptian Embassy \cite{zhou_side-channel_nodate}. To reduce the computational effort, they listened to the setting of the device through a microphone placed close to a window, and thus deduced the position of some of the machine's rotors.
By their nature TDX can not protect against all Side-Channel attacks, but importantly, it is effective against all currently known forms of the "Spectre-Family" attacks. Explicitly outside the scope of the TDX threat model are computation-time and power-analysis attacks, as well as studies from memory access patterns.

\myparagraph{Vulnerable Applications}

The guest Operating System is part of the \Gls{TCB} but is not under the control of the TDX creators or developers. Any issue within an \Gls{OS} can cause issues within a \Gls{TD}. This would also happen with a server that is hosted on your own premises and not  and is thus not being protected against. Similarly, any security issue within an outward facing application within a \Gls{TD} is a security issue that can not be fixed by using TDX. This means that an application inside a TD is at most as secure as the application itself. 


\subsection{TDX Attestation}
\label{TDX attestation}
Hardware attestation is a process that establishes the integrity and trustworthiness of hardware components in a computing system. It involves verifying the identity and expected behavior of hardware components connected to the motherboard. It can be used standalone, but also in connection with software attestation. Software attestation can measure parts, or the entirety, of the memory and report these measurements to a remote party, which can then verify those measurements \cite{stumpf}. This verification is done by checking identification codes and comparing them to expected values. Software attestation can use hardware attestation as an anchor for its chain of trust. Attestation is necessary to establish trust for a challenger that the used hardware is as expected. Hardware and software attestation can be used to establish trust into a remote platform for a challenger.
The TD attestation protocol involves six key entities: the Quoting Enclave (QE), the host Virtual Machine Manager (VMM), the guest trust domain (TD), the Intel TDX module, the CPU hardware and the challenger. The QE is an Intel-provided enclave that signs the report body after its successful verification to create a remotely verifiable quote. The VMM is an untrusted hypervisor that manages Virtual Machines. The guest TD is an enhanced Virtual Machine that is initialized and measured at boot time. The TDX module is Intel-provided software that is signed by Intel and manages the interaction between the TD and the VMM. It is also part of the attestation metadata. The CPU hardware generates and verifies the report. Lastly, the challenger (also known as the relying party) is the remote party that performs the attestation verification.


\label{Pre-Attestation setup}
Before any attestation challenge request can be answered the platform itself must be registered at the Intel Provisioning Certification Service \cite{intel_corporation_dcap_2024-1}. This is done by creating an asymmetric shared platform key using the hardware-keys of the different CPUs on the platform. It is shared between the different CPUs. This shared platform key is then encrypted using those hardware-keys and safed on the NVRAM of the BIOS. After booting the host platform accesses a platform certificate via the Provisioning Certification Enclave (PCE), which is signed using the shared key created previously. The host then has to register the platform at the Provisioning Certification Service (PCS). During this registration Intel checks the validity of the platform manifest. If verified the PCS returns a certificate for the platform key to the PCE \cite{cheng_intel_2023}. The result can be cached locally using an implementation of the Intel provisioning certification caching service \cite{caching_service}. Next the Trusted Domain Quoting Enclave (TD QE) needs to have its own key signed. The TD QE is an an SGX-enclave running on the Host, which later signs the TD quotes during the TDX attestation. The PCE now needs to establish that the TD QE is running on the same platform as it is, this is done via local attestation. Secure connection between two SGX enclaves is explained in more depth in \cite{intel_corporation_migration_spec_2023}. The local attestation is shown in \cref{fig:local-attestation}. The steps needed are as follows:
\newcommand\setItemnumber[1]{\setcounter{enumi}{\numexpr#1-1\relax}}
\begin{enumerate}
    \item The TD QE generations an asymmetric key-pair called attestation key (AK)
    \item The TD QE sends a request for local evidence to the CPU hardware. This request includes the hash of the public part of the previously generated AK combined with any additional user provided authentication data. Intel does not provide information if this is filled. Additionally the TD QE has to specify the verifying enclave as the PCE.
    \item The CPU packs this information and claims about the TD QE and MACs it using the Report Key (RK) into the QEReport.
    \item The QEReport is returned to the TD QE
    \item The TD QE forwards the QEReport and its public AK to the PCE
    \item[6. \& 7.] The PCE requests the RK from the CPU hardware
\setItemnumber{8}
    \item The PCE verifies the evidence based on the policy set by the PCE owner. This includes at least verification of the MAC over the QE Report Body and the hash of the public AK.
    \item The PCE then signs the claims using the PC Key and returns this to the TD QE. According to \cite{scarlata_supporting_nodate} the so-called AK cert is a cert-like structure identifying the QE and the Attestation Key, they do not further specify how the certificate is made up but according to \cite{sardar_formal_spec_ARM_2024} it contains the public AK, the QEReportBody and the signature PCKsig over the QEReportBody.
    
\end{enumerate}

\begin{figure}
\centering
\includegraphics[width=0.9\textwidth]{figures/Local Attestation.png}
\caption{Intel TDX local attestation flow diagram. Text above the arrow represents data being sent, text below function calls. The Host VMM in red is untrusted. The challenger is shown to have the same entities as \cref{fig:QuoteGeneration}. Adapted from \cite{sardar_formal_2023}}
\label{fig:local-attestation}
\end{figure}

Complementary to the three requirements required by the Confidential Computing Consortium, in attestation generally according to \cite{sardar_demystifying_2021}, adapted to fit with TDX attestation vocabulary, the four most interesting properties are:

\label{FourProperties}
\textit{Integrity}: Claims within the TD quote, represent the present condition of the TD and contain verifiable components such as identity fields. Therefore, it is crucial to prevent an adversary from altering claims while they are being transported from the TD to the challenger. Typically, the integrity of claims is safeguarded through digital signatures utilizing an Attestation Key. If integrity holds, then it means that the adversary cannot modify claims inside the TD quote without being detected.

\textit{Freshness}: Freshness means that the TD Quote is the latest created quote to a specific request. The freshness of Evidence is crucial because otherwise an attacker potentially has the ability to replay authentic Evidence from a previous session while also altering the state of the TD.


\textit{Authentication}: The challenger must ensure it communicates with the intended TD. Informally, if the Verifier receives the public key of the TD, this uniquely matches the public key generated within the TD. This implies that the adversary does not have access to keys or secrets related to the attestation shared between the TD and the challenger, which is called secrecy.

Hardware attestation establishes trust in the hardware, but not in the software being executed. The Attestation does however generate a quote that contains additional information on the TD and the software being executed. Trusting the Quote does not mean trust with the TD is established. This establishes the environment of the TD, mainly the specific hardware and the TDX Module. With the additional information about the VM and the Image in the Quote verification might be possible but the implementation of that is left to the software developer. More information on this can be found in \cref{Identity}.
\begin{figure}
\centering
\includegraphics[width=\textwidth]{figures/Attestation-einfach.png}
\caption{A simplified SGX-based TD attestation flow. Adapted from \cite[p.~111]{noauthor_tdx-module-10-public-specpdf_nodate}}
\label{fig:EasyAttestation}
\end{figure}
Figure \ref{fig:EasyAttestation} shows a rudimentary TDX attestation flow with trusted and untrusted entities as well as their boundaries. The challenger requests a quote from the guest TD for challenge. He can include a one-time key, called nonce to prevent replay attacks. A quote is all the information necessary to attest to a specific set of hardware. After receiving the request the guest TD can supply additional runtime data, which is important to verify the identity of the TD, and will then send a request via a character device to the TDX module. This report contains data the TDX module holds about the TD, most importantly measurements relating to the creation of the TD, and the data that was supplied by the TD. To prevent a data breach during the data transfer via the untrusted Host VMM the CPU has a Message Authentication Key (MAC) from the TD QE, which it uses to MAC the TDReport. The TD QE now calls the CPU again to verify the integrity of the TDReport.
To create a chain of trust, which can only be accessed via Intel, the Attestation key was previously signed by the PCE key, which was signed by Intel. 

\begin{figure}
\centering
\includegraphics[width=\textwidth]{figures/Attestation Diagram.png}
\caption{Intel TDX Attestation flow diagram. Text above the arrow represents data being sent, text below function calls. The Host VMM in red is untrusted. The PCE is shown to have the same layout as \cref{fig:local-attestation}. Adapted from \cite{sardar_demystifying_2021}}.
\label{fig:QuoteGeneration}
\end{figure}
The following will now explain each step in more technical detail, while using Figure \ref{fig:QuoteGeneration} as reference. 
TDX includes a set of major functions explained here. \textit{Sign} represents the Elliptic Curve Digital Signature Algorithm signature over the message with the specified signature key. 
Calculating the \textit{hash} means computing the SHA384 hash of the input. \textit{Hmac} is HMAC\_SHA256 of the message with the specified key. More information on hmac can be found in \cite{hmac_keying_1996}. The message is not extractable if hmac is considered a pseudorandom function, which in practice appears to be true\cite{bellare_new_2006}. The TDK key-pair is sometimes also called TD key-pair.
\begin{enumerate}
\item The challenger initiates the attestation process by sending a challenge request to the Guest TD. This can include a nonce to prevent replay attacks\cite{sardar_formal_2023}.
\item The TD calls the TDG.MR.REPORT function to initiate the attestation on the system. It can specify any additional data in the REPORTDATA field, called rdata from here. On Linux this can be done via the character device tdx\_guest. This triggers a TDExit as shown in \cref{fig:seamFigure}
\item TDXM assembles TD information data structure tdi from Trust Domain Control Structure (TCDS) and computes its SHA384 hash tdih. The TCDS contains the following information:
\begin{itemize}
    \item Fields designed to control the TD operation as a whole (e.g., a counter of the number of VCPUs currently running). 
    \item Fields designed to control the execution control of the TD (debugability, CPU features available to the TD, etc.). 
    \item Registers filled with static and runtime measurements. 
    \item EPTP: as designed, a pointer (HPA) to the TD’s secure Extended Page Table (EPT) root page and EPT attributes
    \item Model Specific Register (MSR) bitmaps, designed to be used by all the TD’s VCPUs. 
    \item A page filled with zeros, designed to be used in cases where the Intel TDX Module needs a read-only constant 0 page encrypted with the TD’s private key.
\end{itemize}
\item TDXM calls the SEAM instruction SEAMREPORT with tdih and rdata
\item[5. \& 6.] CPU generates the SEAMREPORT, smr with a MAC block of hashes and the report data called rms and the TDX module measurements called tcbi. They are respectively red and purple in \cref{fig:tdr}. It then returns that report to the TDX module. This importantly contains measurements about the TDX module and about the TDX module signer, in this case Intel.
\setItemnumber{7}
\item TDXM builds the TDRreport tdr with the smr, a reserved block of memory, and the tdi created in step 3. Content for the entire tdr can be seen in Figure \ref{fig:tdr}
\begin{figure}
\centering
\includegraphics[width=\textwidth]{figures/tdr.png}
\caption{An overview of the component of the TDREPORT taken from \cite{sardar_demystifying_2021}}
\label{fig:tdr}
\end{figure}
\item TDXM sends tdr to the Guest TD
\item[9 \& 10.]The Guest TD sends tdr to the TD Quoting enclave, via the untrusted VMM
\setItemnumber{11}
\item TD QE verifies the hashes in report
\item Rms (part of smr, red in Fig \ref{fig:tdr}) is used as an argument in the 
ENCLU[EVERIFYREPORT2] cpu function. 
\item[13. \& 14.] CPU performs the verification of rms and returns the result to the TD QE. The verification consists of three main steps: 
\begin{enumerate}
\item verify that the header rtyp in the report is correct, 
\item verify that the CPUSVN is a valid value, and 
\item compute the MAC over the fields in the report body rptbody using the MAC key MACkey, and verify that the computed MAC matches the value in the field mac of the received report (represented as receivedMAC)
\end{enumerate}
\setItemnumber{15}
\item TD QE replaces the MAC in the rms with <rptbody, (sign(AK, rptbody))> to create a complete quote.
\item[16. \& 17.] Quote is sent to TD, VMM is untrusted → Sent by public channel
\setItemnumber{18}
\item TD sends quote to relying party(RP)
\end{enumerate}
The challenger can now verify the Signature on the Quotebody by going back the chain of trust rooted at Intel. This verification is based on the Data Center Attestation Primitives (DCAP) which will be explored later on in \ref{Security Analysis}. After verification the challenger can now send a secret encrypted with the public key of the TD to the TD establishing a common secret for further secure communication. If the TD did not include a public key in its quote section \cref{Establishing_a_secure_connection} contains further information on how to establish a secure connection. According to \cite{sardar_formal_2023} using ProVerif, this ensures integrity for the tcbi, tdi and rdata. It also ensures freshness and secrecy. Authentication does old as long as secure communication via TLS or similar is not established, which will be looked at in \cref{Establishing_a_secure_connection}.Ihc gu

\section{Related Work}
TDX has been looked at a fair amount of times already. In particular, a nearly complete security analysis of its hardware and the low-level TDX software by Aktas et al. at Google \cite{aktas_intel_nodate}. They limited their analysis to just the host-side without looking at TD user or developer issues down the line. Similarly, Sardar et al. formally verified TDX attestation in \cite{sardar_demystifying_2021}. They looked at the theoretical security of a perfect implementation. Knauth et al. introduced a way to create a secure channel using Intel SGX \cite{knauth_integrating_2019}, this will be used as the basis to establishing a secure channel to the TD in this thesis. Cheng et al. briefly mention this as well, but their focus was more generally on the TDX architecture \cite{cheng_intel_2023}. This thesis explores the feasibility of those for an average user, as well as its security assumptions and pitfalls. Lefeuvre et al. have looked at difficulties and issues with safe and confidential I/O \cite{lefeuvre_towards_2023}, this will only be briefly touched upon in this thesis. Delignat-Lavaud et al. looked at issues pertaining the TCB of confidential services. Their information was used in this thesis to look at the feasibility of using CC for the average user. Their consensus was picked up in this thesis that while CC is great in theory, the practice is lacking.



\chapter{Methodology}

- Mad vs std dev
- Performance testing how to
- How to rank values / Zangenmeister?

%% LaTeX2e class for student theses
%% sections/content.tex
%% 
%% Karlsruhe Institute of Technology
%% Institute for Program Structures and Data Organization
%% Chair for Software Design and Quality (SDQ)
%%
%% Dr.-Ing. Erik Burger
%% burger@kit.edu
%%
%% Version 1.4, 2023-06-19

\chapter{Security Analysis}
\label{Security Analysis}

This section contains an overview of threats and how, if so, TDX protects against them. This analysis is by no means comprehensive, but meant as an overview. In addition ways to establish a secure connection to a TD are discussed.
\section{The Threat Model}


\subsection{Threads TDX does not protect against}

\subsubsection{Side-Channel Attacks}

Side-Channel Attacks are attacks that take advantage of unintended output channels of a system. The first one according to \cite{zhou_side-channel_nodate} which in turn references Wright et Al., was when in 1965 MI5 tried to decipher a cryptography device used in the Egyptian Embassy. To reduce the computational effort, they listened to the setting of the device through a microphone placed close to a window, and thus deduced the position of some of the machine's rotors.
By their nature TDX can not protect against all Side-Channel attacks, but importantly, it is effective against all currently known forms of the "Spectre-Family" attacks. Explicitly outside the scope of TDX are computation-time and power-analysis attacks, as well as power-glitches and physical fault injections.

\subsection{Vulnerable Applications}

The Operating System is part of the \Gls{TCB} but is not under the control of the TDX creators or developers. Any issue within an \Gls{OS} can cause issues within a \Gls{TD}. This would also happen with a server that is hosted by yourself and not offshore and is thus not being protected against. Similarly, any security issue within an outward facing application within a \Gls{TD} is a security issue that can not be fixed by using TDX. This means that an application inside a TD is only as secure as the application itself. 

The threat model does not address any threats made possible by the TDX guest userspace directly using TDVMCALL hypercalls (through the TDX-module) and shared memory for IO exposed to an untrusted host/VMM. If TDX guest userspace enables debug or test tools that can read memory-mapped I/O (MMIO) or PCI configuration space without validating the input from untrusted host/VMM, it opens up the possibility of many more attacks. In addition malicious input from, for example,  PortIO, SharedMemory or KVM Hypercalls to non-robust device drivers can lead to threats from inside the TD, although Intel has provided mitigation measures against those \cite{linux-guest-hardening}.


Intel has published a white paper \cite{noauthor_tdx-whitepaper-february2022pdf_nodate} which contains an overview of what TDX is designed to mitigate. The TCB is very conservative only containing 
\begin{itemize}
    \item Intel TDX module, with Persistent Seam Loader (P-SEAMLDR)
    \item Intel Authenticated Code Modules (ACM)
    \item TD Quoting Enclave (SGX)
    \item Intel CPU Hardware
\end{itemize}
All other system components are outside of the TCB, including the BIOS, SMM, host OS, and the VMM. Some physical attacks such as cold-boot and DRAM traffic are also supposedly protected against. This threat model also assumes the KVM/Qemu to be the hypervisor running the protected TDX guest \cite{aktas_intel_nodate}.

\subsection{Adversarial Goals}
An attacker targeting TDX or a TD directly can focus on different components, depending on what their goals are. In general, they would be interested in leaking sensitive data or disrupting the expected behavior of a TD.

\myparagraph{Leaking TD Secrets}

Once the TD has been verified via attestation, the third party owner of the TD is expected to provide confidential information to the TD. This could take the form of cryptographic keys, private user data, or similar. TDX aims to isolate this information from any malicious third party. If any attacks exist the Host, neighbouring TDs or someone with hardware access could potentially extract this information.

\myparagraph{Manipulating TD Behavior}

An adversary may be looking to change the behavior of a victim TD in various ways. This could be done through direct memory or register corruption, which would cause unexpected behavior or control of execution. As previously discussed the VMM can change some memory mappings the TD uses, which could lead to changes in the behavior of the application. More subtle ways to do this involve changing the I/O to the TD or even abuse the scheduler for this. 

\myparagraph{Host Denial-of-Service}

An adversary may aim to limit the availability of a cloud provider by obstructing the scheduling of other workloads (TDs, VMs) on a machine. This can also be achieved by abusing the scheduler. Additional attacks can vary in their severity. For instance, a bug in a TD may lead to its shutdown, while a bug in the TDX module could necessitate an immediate halt of all TDs on the machine. Some memory states can cause unrecoverable machine checks which necessitates a full power cycle to recover\cite{aktas_intel_nodate}.

\subsection{Attack Vectors and Vulnerabilities}

The TCB for Intel TDX has a restricted number of system components, which creates only a few opportunities for adversaries to breach system security. Moreover, when a malicious Virtual Machine Manager (VMM) and a malicious TD work together, they can exploit complex scenarios that the system designers may not have foreseen. The following will give a quick summary of the attack vectors from \cite{aktas_intel_nodate} and then give an additional vector that is the main focus of this thesis.  


\myparagraph{MCHECK, exiting the Bios safely}

With TDX and its predecessor SGX the BIOS is situated entirely outside the TCB. This results in the system requiring a mechanism to ensure that the BIOS has configured all security-sensitive settings to be within an acceptable range. Intel has developed MCHECK, not to be confused with the malloc debugging tool, to provide this insurance and provide results to TDX in a trusted way.  MCHECK is embedded within the CPU microcode update file, which is encrypted and signed by Intel. This ensures that only Intel can execute microcode updates but it makes this a completely opaque security validation. MCHECK code is not made public by Intel. According to Intel MCHECK verifies \guillemotright all logical cpus to ensure they meet TDX's security and certain functionality requirements \guillemotleft\cite[p.~50]{noauthor_tdx-module-10-public-specpdf_nodate} and then provides an overview of general hardware information to Non-Persistant SEAM Loader (NP-SEAMLDR) and P-SEAMLDR. 

\myparagraph{NP-SEAMLDR and P-SEAMLDR}

Since the BIOS code for the startup is outside of the \gls{TCB}, there needs to be a method to establish a root of trust on which the rest of the TDX infrastructure can be loaded. For TDX the already existing TXT and Authenticated Code Modules were extended to create a new Authenticated Code Module name NP-SEAMLDR \cite{intel_corporation_intel_TXT_nodate}. The OS loads NP-SEAMLDR which validates the system configuration and installs P-SEAMLDR into the \Gls{SEAMRR} memory regions and then returns control to the OS. NP-SEAMLDR prevents calls from the BIOS progressing beyond the entry code, meaning the BIOS can not execute code while after this state. All guest VMs and TDs have to trigger a VM exit for all \Gls{GETSEC} instructions. This leaves a very limited BIOS, the OS/VMM and SMM as attack vectors into NP-SEAMLDR. An issue, in a previous version, found by Aktas et Al. allowed an attacker to have free control over the Instruction Pointer while in privileged AC mode thus allowing an attacker to execute arbitrary commands \cite{aktas_intel_nodate}. This is fixed now but arbitrary command execution would completely subvert any trust in the \Gls{TD} without being visible to a challenger. P-SEAMLDR after it was installed then installs the main TDX module serially on all logical processors. This creates a chain of trust from MCHECK to the TDX Module and then later on to the TD \cite{intel_corporation_intel_seam_2023}. 

\myparagraph{\Gls{TDX Module}}

The TDX module is the central privileged software component for running TDs. P-SEAMLDR installs the module and runs it in SEAM Root Mode giving it full access to the host OS and all TDs. The TDX module could be attacked either from a malicious TD or a malicious Host. A malicious TD is by design not able to affect other TDs so this attack vector is not as important. Taking over the TDX module or being able to execute any code would make any attestation or security guarantee useless. The security attestation only covers if the TDX module is executed correctly and if the code is signed by Intel. If this code has exploitable faults, the attestation does nothing to help the user.

\myparagraph{Outside connection to the TD}

The last major interface is the outside connection to the TD. Here two different problems arise. Firstly transport data security. Making sure that data is not compromised in transit is technically not part of the threats TDX wants to protect against and it does not do that but it will be a major concern of this thesis. The second issue that comes up is making sure that communication is done with the correct TD. TD identity checking is explicitly outside the scope of TDX but it can help with that and this will be another concern in this thesis. The TDX attestation only guarantees that somewhere a TD runs on certified hardware with a signed TDX Module version. Additional information the Quote can supply will be needed to identify a TD. An exemplary man-in-the-middle attack is shown in \ref{fig:man_in_the_middle}.
\begin{figure}
   \centering
       \includegraphics[width=.75\textwidth]{figures/Man-In-The-Middle.png} 
 \caption{A simple man-in-the-middle attack against a TD. The attacker reroutes the challenge to an unsecured VM, which in turn requests a quote from a different TD}
 \label{fig:man_in_the_middle}
\end{figure}
Provided that security promises within the previously mentioned parts of the TCB are solid, connecting to and verifying the identity of a \Gls{TD} is the last step towards secure computation with user Input.

\section{Connecting to the correct TD}

This section proposes three different ways to establish a secure connection, which get increasingly more complex, while reducing security assumptions and increasing adherence to best practices.

\myparagraph{Verifying the TD identity}

\label{Identity}
Similarly to SGX, with TDX the quote contains information on the startup environment of the TD. The Measurement of Trust Domain (MRTD) registers contain information on the initial state of the TD immediately after startup \cite{intel_corporation_dcap_2024-1}. The Runtime Measurement Registers (RTMR) are written during the runtime, Intel recommends writing information about the virtual firmware configuration in RTMR[0] and information about the OS kernel and boot parameters or more importantly kernel parameters in RTMR[1]. Additionally the Quote can contain informal data in the \textbf{REPORTDATA} field, which could for example contain hashes about the current state of the filesystem. The developer has to compare the expected state of the TD with the real state themself. As explained in \ref{sec:Foundations:ConfComputing} Linux can leverage IMA to measure the integrity of its file system. IMA can be used alongside ARM Trustzone for runtime integrity measurement for applications located in edge appliances and client appliances \cite{song_tz-ima_arm_2022} and has subsequently also been integrated in Intel TDX. This requires changes to the Linux Kernel to use the correct runtime measurement registers, as long as vTPM is not supported by TDX. The records of the event logs related to the list of measurements will be kept in the Confidential Computing Event Log (CCEL) table \cite{haidong_xia_runtime_integrity_measurement_2024}. 

\subsection{Establishing a secure connection to the TD}

\label{Establishing_a_secure_connection}

\myparagraph{Simple SSH connection to a TD}

\label{SSHConnection}
After creation the TD contains a previously supplied public key. This is generally put into the image prior to boot and if it is the only key present and password authentication is deactivated, the TD will decline connection attempts by anyone without the corresponding private key. This being the only key present can be verified via measurements in the TD Quote as discussed previously. If possible the startup measurements should be compared to a selfbuild image containing just the users public key. If the user is the only person being able to access this specific TD then this means that it is no longer possible to fake the man-in-the-middle attack in Figure \ref{fig:man_in_the_middle} because the attacker would not have access to a TD that can create the correct quote. This form of secure communication relies on the fact that only one public key is in the TD and thus the initial communication channel is secure. If somehow the initial communication channel can be compromised this is no longer secure. It also depends on the user to verify the TDs identity upon connecting and also for subsequent connection attempts that the fingerprint of the TDs can not be recreated for transcript collision attacks \cite{bhargavan_transcript_2016}.

\myparagraph{SSL connection to a TD}

According to \cite{cheng_intel_2023} TDX and SGX can establish secure channels via SSL in a similar way:
In a typical scenario, when a client negotiates a secure channel with a server running in a TD, it aims to ensure a connection with a server that has been properly instantiated. The server, acting as an attester, generates a pair of ephemeral public and private keys. It calculates the hash of the public key and creates a TD report that includes this hash as the \textbf{REPORTDATA}. The server then requests a quote of the report and generates a self-signed certificate with the quote embedded in it. This self-signed certificate is provided as the server certificate in the TLS handshake protocol. Upon receiving the server certificate, the client, acting as the challenger, verifies the signatures on the certificate and validates the embedded quote, including the measurements. The client also checks if the quote includes the hash of the public key, as this associates the key with the TD. This contains the assumption that the client can verify the TD identity from the Quote, which is not given. When establishing a secure channel, both the client and server can assume the roles of attester and verifier. This enables endpoints running in TDs to authenticate each other mutually by validating TDs\cite{knauth_integrating_2019}. Issues arising from self-signed certificates, the certificate size and similar are discussed as well, with all of them  having solutions. This form of secure communication relies on certificate authorities and can thus also work if the initial communication channel is compromised. Although the latter can mean that the entire TD can be compromised which invalidates other security assumptions. They thus have similar security assumptions and guarantees. This is also the way Intel does TD migration in their  TD migration guide\cite{intel_corporation_td_migration_design_2023} which in turn also relies on \cite{knauth_integrating_2019}.
Lastly an add-on to the previous method of establishing a secure connection is the usage of a complete application in the image, that does only expose certain endpoints to the web. Using proper authorisation against those endpoints, makes unwarranted access impossible. Additionally, having the application immediately on startup create the necessary certificate from the previous paragraph, prevents attacks on the initial connection as well. This is Intels recommended way of using TDX and starting a TD. 

\myparagraph{Using Initramfs and encrypted disk images}

Using a custom Kernel with a custom Initramfs that mounts an encrypted disk and contains TDX attestation capabilities as well as a predetermined public SSH key, it is possible to establish a secure connection as well. This relies on the same mechanisms as \ref{SSHConnection} but gives the additional guarantee, that the Kernel loaded has to be the one supplied by the user. It can also safeguard sensitive data inside the image from unauthorized access. Upon startup the TD will pause the start prior after the Kernel has been loaded. The Initramfs will be loaded and active here as well. The challenger can now use its private key to the public ssh key inside the Initramfs to establish a secure connection. The challenger can now create a TD quote containing the Kernel measurements. Verifying the kernel measurements inside the TD quote and can then supply the key to the encrypted disk image to continue booting. It is also possible to use the Storage-Volume-Key-Location ACPI Table in the virtual firmware for this \cite{uefi_forum_inc_acpi_docu_2022}. Intel recommends implementation of these tables in their design guide \cite{intel_corporation_tdx-virtual-firmware-design-guide-rev-004-20231206pdf_2023}. If disk encryption is used, LUKS provides the ability to use disk encryption with authentication. It is now guaranteed that the image inside the encrypted disk image is now executed only once and is running inside a TD. Choosing a secure channel to this TD depends on the code inside. Having an additional SSH key inside, as explained previously, can work.

\section{Attestation Verification}

There are generally four ways how attestation can be done that have their pros and cons. They are outlined in \cref{tab:AttestationVerification}. 
\begin{table}
\centering
\resizebox{0.9\textwidth}{!}{%
\begin{tabular}{ m{0.3\textwidth} m{0.2\textwidth} m{0.2\textwidth} m{0.2\textwidth} m{0.2\textwidth}}
\toprule
& Cloud Provider Attestation Service & Application Vendor Attestation Service & Independent Trust Service (E.g. Intel Trust Authority) & Build-Your-Own Service with DCAP \\
\midrule
Seperation of responsibilities between verifier and infrastructure provider & No & Yes & Yes & Yes \\
\midrule
Consistency across SGX and TDX & Yes, if both are offered & Yes, if both are supported & Yes & Yes \\
\midrule
Consistent service across on-prem, hybrid, multi-cloud, and edge deployments & No & Possible but unlikely or limited & Yes & Yes\\
\midrule
Development Effort & Low & Low & Low & Medium \\
\bottomrule
\end{tabular}
}
\caption{Overview of four different Attestation Verification methods}
\label{tab:AttestationVerification}
\end{table}
This thesis will focus on Cloud Provider Attestation Service, and more specifically Microsoft Azure Attesation (MAA). The security assumptions and guarantees for this will then be compared to a Build-Your-Own Service using DCAP. MAA also has to use DCAP to generate and verify quotes.

\subsection{Cloud Provider Attestation Service}
 Microsoft, as do most other cloud providers, offers an attestation verification service hosted by them. They also offer an implementation of Intel DCAP, which can be used to generate a quote and automatically verify it via MAA or Intel Trust Authority. The latter was not available for testing.

\myparagraph{Microsoft Azure Attestation}

This section will explain in detail how MAA can be used, the pitfalls and issues with it. Microsoft supplies a tutorial to create a TD VM at  \cite{chasecrum_github_create_2024}, which does not create a TD, that has the characteristics discussed in \cref{TDX Architecture}. Issues with the used TD as well as the firmware are explored in \ref{Issues-with-azure-td}. Following the instructions provided by Azure, it is also not explained how to choose between AMD SEV-SNP and Intel TDX. This decision is entirely up to the chosen VM size which can only be inferred by looking at press releases pertaining Intel TDX and the newly released VM sizes. The DC\textbf{e}sv5, DC\textbf{e}dsv5 and EC\textbf{e}sv5, EC\textbf{e}dsv5 use Intel Xeon CPUs, while DC\textbf{a}sv5, DC\textbf{a}dsv5 and EC\textbf{a}sv5, EC\textbf{a}dsv5 use AMD Threadripper CPUs.The difference between D-instances and E-instances is the ratio between VCPU cores and memory, with E-instances having more memory per core. With the VM created there are a couple of ways to create a TD Quote for attestation: Using an Intel-supplied or a self-written implementation of Intel DCAP and the Azure confidential-computing-cvm-guest-attestation library \cite{microsoft_corporation_azureconfidential-computing-cvm-guest-attestation_nodate}. Building the tdx-attestation app and using the supplied maa\_config file, which does only contain three settings, succeeds in creating a TD quote and then returns a JSON Web Token from the Microsoft Azure attestation Service. The entire decoded token can be found in \ref{jwt} but here we will focus on some parts of the td quote body. \cite{intel_corporation_dcap_2024-1} contains short explanation for all registers, which will be explained further. We know that this has to be TDX 1.5 because tdx\_mrseam is not filled with 0, which would indicate TDX 1.0. The TEE\_TCB\_INFO struct contains two measurements, which are expected to be non-zero: MRSEAM and TEE\_TCB\_SVN, one that has to be 0 with TDX: MRSIGNERSEAM and one that can be either: TD\_ATTRIBUTES. These all fit. Intels recommendation for runtime-measurement registers tdx\_rtmr0 and tdx\_rtmr1 are as follows: RTMR0 stores the measurements for TD Virtual Firmware, these are influenced by tdvm launch parameters, such as memory size. Most importantly this is locked in during startup and thus not user controlled in a cloud environment, although those can be overwritten, those changes would be present in the ACPI CC log table \cite{uefi_forum_inc_acpi_docu_2022}. In the excerpt returned by MAA this is filled with 0. RTMR1 should have two different purposes, depending on the boot option for the VM . With a direct boot it stores the kernel measurements and cmdline, that is passed to the kernel. With a grub boot it stores the grub measurements \cite{intel_corporation_tdx-virtual-firmware-design-guide-rev-004-20231206pdf_2023}.  Both of those measurements are once again locked-in during the VM startup and out of control of the user. The register returned by MAA is once again only filled with 0, which is in itself not a violation of any standard but odd. The virtual firmware Intel offers for use does follow these instructions, meaning the firmware in use is a custom one by Microsoft which overwrites these best-practices. Especially RTMR1 with its Kernel measurements is invaluable to the identity verification measurements provided in \ref{Identity}. 

\begin{lstlisting}[language=jsonmain,caption={TDX generated part of an MAA quote},captionpos=b]
  "tdx_mrconfigid": "000000000000000000000000000000000000000000000000000000000000000000000000000000000000000000000000",
  "tdx_mrowner": "000000000000000000000000000000000000000000000000000000000000000000000000000000000000000000000000",
  "tdx_mrownerconfig": "000000000000000000000000000000000000000000000000000000000000000000000000000000000000000000000000",
  "tdx_mrseam": "360304d34a16aace0a18e09ad2d07d2b9fd3c174378e5bf108388079827f89ff62acc5f8c473dd40706324834e202946",
  "tdx_mrsignerseam": "000000000000000000000000000000000000000000000000000000000000000000000000000000000000000000000000",
  "tdx_mrtd": "024a32b070383331181619fa387cb4d55d1e38879f989933055ccad5bc2db795d1737b66205949d15469dc8c1ba7ab7b",
  "tdx_report_data": "c90f98ba8ab80c7b442b6b8eb30af54e0508077b11adb525af6dfbcc8714e52a0000000000000000000000000000000000000000000000000000000000000000",
  "tdx_rtmr0": "000000000000000000000000000000000000000000000000000000000000000000000000000000000000000000000000",
  "tdx_rtmr1": "000000000000000000000000000000000000000000000000000000000000000000000000000000000000000000000000",
  "tdx_rtmr2": "000000000000000000000000000000000000000000000000000000000000000000000000000000000000000000000000",
  "tdx_rtmr3": "000000000000000000000000000000000000000000000000000000000000000000000000000000000000000000000000",
  "tdx_seam_attributes": "0000000000000000",
  "tdx_seamsvn": 258,
  "tdx_td_attributes": "0000000000000000",
  "tdx_td_attributes_debug": false,
  "tdx_td_attributes_key_locker": false,
  "tdx_td_attributes_perfmon": false,
  "tdx_td_attributes_protection_keys": false,
  "tdx_td_attributes_septve_disable": false,
  "tdx_tee_tcb_svn": "02010600000000000000000000000000",
  "tdx_xfam": "e718060000000000"
\end{lstlisting}
\label{td_quote}

\myparagraph{Issues with the Image and Firmware used in the Azure TD}

\label{Issues-with-azure-td}
For a TD to be present three different properties are necessary, although they are themselfs not sufficient to proof connection to a TD. In the file /proc/cpuinfo there needs to be a flag for tdx\_guest, this was false. The CC Event Log ACPI Table needs to contain CC Type 2 for TDX - this table was not even present, which is contrary to best efi best practices \cite{uefi_forum_inc_acpi_docu_2022}. Information from this table are also necessary for TD validation in general. The information from this would have also been present in the rtmr0 register. Lastly a pipe or character device called tdx\_guest, tdx-attestation or tdx-guest depending on the version should be present. These function as an interface to retrieve TDX-guest specific information\cite{linux_kernel_development_community_tdx_2024}. It is used to retrieve the TDREPORT for the attestation, covering steps 2 and 8 in Fig\ref{fig:QuoteGeneration}. This was also not present. This means that the Guest VM was neither a TDVM, nor is it compliant with other confidential computing VM standards. This itself does not make MAA hardware attestation impossible, but makes it much more difficult to implement your own attestation. It appears that Microsoft recreated the tdx\_guest interface somewhere but does not tell the user how it was done or if it was done exactly the same way. The User needs to recreate this pipe manually if they want to implement their own attestation. These pipes are not interchangeable between VMs of the same operating system. All of these changes go against the recommendations by the EFI specifications as well as Intels recommendations for TD implementations.
Verifying MAAs implementation is not possible as \guillemotright"Source code is [only] available for government customers via the Microsoft Code Center Premium Tool \guillemotleft \cite{dan_mabee_azure_attestation_2023}. It is not possible to supply your own image and the provided images by Azure do not have public hashes. Using the TDX IMA integration discussed in \ref{Identity} is also not possible with Azure in general.
The Azure attestation token is signed with a self-signed certificate, which was retrieved from \url{https://attestationprovidername.attest.azure.net/certs}, the signature does match. The same certificates contains the SGX quote from the azure attestation enclave, which was verified via the valid implementation of DCAP located at \cite{microsoft_corporation_azure-samplesmicrosoft-azure-attestation_nodate}. This confirms just that azure attestation runs inside a functioning SGX enclave. Additionally it is possible to validate the binding of the azure attestation SGX key with the key that signed the attestation token via the hash of the public key in the reportdata field of the quote. Lastly the Azure Attestation code measurements have to be validated. This was not done for this thesis, as this relies on the azure attestation teams notification for changes to MRSIGNER, which was deemed to be outside of the scope of this thesis.

Hosting your VMs bare metal on your own hardware will offer performance increases similar to the ones outlined in \ref{performance} as you can disable TDX completely. With security guarantees similar or better to the ones that TDX can offer.

\subsection{Intel Trust Authority}


\chapter{Performance Analysis}

\label{performance}

\section{Setting up a TDX VM on a new machine}
\label{ch:SettingUpTDX}
Intel supplies documents for a quick setup of TDX machines and TDs. In this section, \cite{noauthor_white_nodate} will be used as a reference. Intel internally supplies additional documents as well and code from there will be replicated here if possible. The Server was supplied as an Intel DevCloud bare-metal machine with two Intel XEON 8490h processors with 60 cores respectively. Setting up Intel DevCloud and Trusted Domain for the first time took the better part of a week. The versioning in the different internal guides was off, which lead to an unusable kernel, which was only fixed by complete reinstall. The TDX-Toolversion used in the end was 2023ww22 from may 2023, which can be found at \cite{intel_corporation_inteltdx-tools_2024}. The guest image was created using the guest image creation tool. The entire BOM can be found in \cref{tab:BOM}

\begin{table}
\centering
\resizebox{\textwidth}{!}{%
\begin{tabular}{ {0.15\textwidth} m{0.2\textwidth} m{0.2\textwidth} m{0.2\textwidth} m{0.2\textwidth} m{0.2\textwidth} m{0.2\textwidth} m{0.2\textwidth}}
\toprule
Component & Kernel & Libvirt & QEMU & Grub2 & OVMF & Shim & amber-cli \\
\midrule
Ubuntu Version & 5.19.17-mvp23v3+6 & 8.6.0-2022.11.17.mvp1 & 7.0.50+mvp9+15 & 2.06-mvp3 & 2023.03.07-stable202302.mvp9 & 15.4-mvp3 & 2023ww21-mvp3 \\
\bottomrule
\end{tabular}
}
\caption{Comparison of different timings of IPEX optimization method}
\label{tab:BOM}
\end{table}
This installs all the necessary kernel components to the images, as well as checking the respective hashes. Next the host kernel has to be patched. This requires downloading the correct packages from the Intel open source directory and then simply creating a patched kernel. Next TDX can be enabled and booted up using the newly patched kernel. To enable the possibility of attestation, the kernel needs Intel SGX Data Center Attestation Primitive (DCAP), which can also be installed from the Intel open source directory or downloaded from Github. We also need an implementation of the Provisioning Certification Caching Service, which Intel provides a reference implementation of, which can also be found in the Intel open source directory. Lastly, the Quote Generation Service contained in the SGX SDK must be installed. Now the Guest TD can be booted up. For this the following QEMU parameters were used:

\begin{itemize}
    \item \todo{QEMU Parameters}
\end{itemize}

\section{Benchmarks}

\subsection{Benchmark and data evaluation methods}

\todo{how benchmarks were run}

In this thesis in general the Median Absolute Deviation opposed to the standard deviation will be used to compare different results. While the standard deviation is the de facto standard in the field of statistics it does not need to be \cite{Gorard}. The main benefit of the standard deviation was in the past its ease of calculation and when drawing from normal-distributed populations \guillemotright the standard deviation5 of their individual mean deviations is 14\% higher than the standard deviations of their individual standard deviations. \guillemotleft \cite{Gorard}
With most results it is generally expected to get something along a normal distribution but this assumption does not appear to be correct with memory benchmarks according to Lemire \url{https://lemire.me/blog/2023/04/06/are-your-memory-bound-benchmarking-timings-normally-distributed/}. With benchmarks having a strict lower limit of 0 we can expect to find right-skewed distributions instead. These large outliers will have a significant impact on the standard deviation, while they are not indicative of an actual slowdown from the measurements in question. Using a rather extreme example the Roberta AMX pipe using a short sentence array benchmark, although most Roberta AMX benchmarks showed these outliers, looks like shown in \cref{fig:histogramm}.
\begin{figure}
\centering
\includegraphics[width=\textwidth]{figures/Histogramm.png}
\caption{}
\label{fig:histogramm}
\end{figure}
A total of 2 out of 270 values were outliers but because some were so far outside the range 1.37s to 2.03s, the std dev was extremely high. Notably the longest run took 6.42 seconds almost 4 times as long as the mean. This one result increases the standard deviation from about 0.11 seconds to about 0.32 seconds. On the other hand the MAD only increases from about 0.03 to 0.04 seconds. The histogram clearly shows a clustering around about  1.65 seconds, which would not be appropriately represented by a standard deviation of 0.32 around the mean of 1.73 seconds.

\todo{General Benchmark results}

\subsection{Benchmark Results}
The Benchmark results using distilBERT-base-uncased show a small performance loss of 2\% from a TD VM to a Non-TD VM. The performance difference fluctuates between the non-TD VM being 3\% slower on the long sentence array transformer pipe to being 5\% faster on the long sentence AMX pipe. Half the benchmarks did not have a t-value that make them significant. For this test the 30 run times were run 3 times for a total of 90 test runs per benchmark. The AMX variant was on average 29\% faster with the differences ranging from 19\% to 50\% faster.
These differences can be seen using MAD and mean and standard deviation, meaning they are not only due to outliers. The first benchmarks had the TD VM about 18\% faster on average. This unexpected result meant running the tests two more times to remove any potential noises in the system. These closed the gap overall but only one suite had the expected outcome of the Non-TD VM running faster. See \ref{sec:appendix:benchmarks:bert} for the full results and \todo{add online archive for .json}.

The benchmark results using RoBERTa show a small performance loss of on average 1\% from a non-TD VM to a TD VM. The performance differs from about 5\% faster to 5\% slower. Each result was checked for significance using Student’s two-sample, two-tailed t-test with an alpha equal to 0.95. The absolute t-values range from 3.79 to 11.28. With 30 runs per test this results in p-values far below 0.01. Every result except 2 was faster on non-TD VMs. The AMX accelerated pipe was faster on the TD vm on the short sentence array and long sentences\todo{why}. So while theses tests are statistically significant the difference in speed is not significant for real-world applications. AMX vs Non-AMX showed a speed increase of about 30\% in this benchmark, this is in line with the expected 1/3 speed increase.
The full Benchmark results can be found in the Appendix \ref{sec:appendix:benchmarks:roberta}.

Both of the previous tests were done using an additional layer using a QEMU VM. Running the same benchmarks on the Host system directly averages an improvement of about 10\%. 
Due to the small margin between non-TD VMs and TD VM it was decided to test what disabling all TDX-related BIOS settings \cite{getting-started} would have as an impact. This showed a significant speedup of on average 24\% using distillbert and 44\% using Roberta. The highest speed-up with 78\% was observed on the short-sentence array using Roberta. The difference was measurably slower using the AMX pipe with the distilbert TDX-enabled AMX pipe on the long-sentence being 5\% faster even. This difference was still inside the standard deviation, so could reasonably be explained with variations in testing. On average the non-AMX benchmarks were 53\% faster using distillbert and 71\% faster using Roberta. Figure \ref{fig:distillbertMADNONAMX} clearly shows the speed difference after disabling TDX in the BIOS. 
\begin{figure}
   \centering
       \includegraphics[width=.95\textwidth]{figures/distillbertMAD.png} 
 \caption{Test}
 \label{fig:distillbertMADNONAMX}
\end{figure}
On the other hand Figure \ref{fig:distillbertMADAMX} shows no measurable speed difference after disabling TDX using the AMX-enabled Benchmarks. 
\begin{figure}
   \centering
       \includegraphics[width=.95\textwidth]{figures/distillbertMADAMX.png} 
 \caption{Test}
 \label{fig:distillbertMADAMX}
\end{figure}
The AMX benchmarks were, using the geometric mean, similar in speed using distillbert, with a difference that was not statistically significant. The AMX benchmarks using Roberta were about 21\% faster \todo{Roberta AMX results}. The overall speed-up was measurably faster than the initial observed AMX speedup, which indicates that this is not due to the omission of AMX in general but something different but having no speed-up using AMX with distillbert was an interesting result. \todo{herausfinden warum}
All prior tests were done using Ubuntu 22.04 and the next were done using Ubuntu 23.10 (from here on Ubuntu 23), because of instability issues. Rerunning the same tests using Ubuntu 23 the observed speed-up from a TD to TDX disabled diminished to just about 3\% with just TDX disabled and then an additional 3\% with memory-encryption disabled as well. These differences were small but the MAD on the measurements was even smaller, as can be seen in  \cref{fig:distillbertAMXUbuntu23}. 
\begin{figure}
   \centering
       \includegraphics[width=.95\textwidth]{figures/inferencedistillberMADUbuntu23AMX.png} 
 \caption{Distillbert Inference times on Ubuntu 23 and Sapphire Rapid CPUs.Note: the apparent decrease in MAD is due to the logarithmic scale}
 \label{fig:distillbertAMXUbuntu23}
\end{figure}
A further examination was conducted on discernible execution time differences using a profiler on Ubuntu 23, but with just an eight percent speed difference, there wasn't much to observe. \todo{Profiler Analyse} Minus the speed overhead of the profiler which was measured to be about 10 seconds, no matter the platform, the execution took between 77 and 85 seconds, with the highest being TDX and memory encryption bypass enabled, which supposedly speeds up calculations, so assuming no impact here, the margin of error has to be at least the difference between memory encryption bypass enabled and disabled benchmarks, which is about two seconds or 2.5 percent. So no useful overall differences can be observed but some internal methods display significant differences. Intels Extension For Pytorch (IPEX) optimizer copy method saw timing increases of up to 400\%, this seems to be entirely due to activating TDX, as without TDX the timings were all within one percent of each other, enabling TDX without memory bypass increased this time by about 100\% and enabling bypass increases this a further 250\%. Total times for this IPEX optimization instruction can be seen in \ref{tab:IpexOpti}. The IPEX "replacement" method for linear transformation which takes most of the calculations shows small speed differences of about 5\% - in line with the rest. Almost all of the remaining speed differences came from basic linear transformation as well as the other methods called by the models. They show speed increases of about 5\% each.
Having a look at the profiler results on the Azure cloud there are no clear results on overall speed but noticeably the IPEX copy and optimize instructions were once again measurably slower on the TD compared to a non-TD VM the optimization took about 20\% longer on the TD. This slow down is almost completely due to the time PyTorches clone function takes, which saw increases of 1000\% in total and per call time. Memory encryption itself had close to no impact on the speed here. Neither did the Memory Encryption Bypass. Looking into the implementation of this function this is not expected. Clone creates shallow copies of a tensor, with the lower hierarchy memory being shared. With both the original and the copy being situated inside the TD, this should have at most the same overhead as raw memory access which is outlined in \ref{tab:MemoryAccessSpeed}.
\begin{table}
\centering
\resizebox{\textwidth}{!}{%
\begin{tabular}{ m{0.2\textwidth} m{0.2\textwidth} m{0.2\textwidth} m{0.2\textwidth} m{0.2\textwidth} m{0.2\textwidth}}
\toprule
& TD with ME Bypass & TD without ME Bypass & No-TD with ME & No-TD with ME Bypass & No Memory Encryption \\
\midrule
IPEX optimize total time & 1.21s & 0.597s & 0.28s & 0.278s & 0.284s \\
\midrule
PyTorch clone method & 1.01s & 0.397s & 0.079s & 0.077s & 0.077s \\
\bottomrule
\end{tabular}
}
\caption{Comparison of different timings of IPEX optimization method}
\label{tab:IpexOpti}
\end{table}

Looking further into Host and Guest implementations of TDX on specific Ubuntu versions was deemed too much for this thesis.

\subsection{Benchmarking TDX parts}

As stated in the previous section there were significant speed differences between having TDX enabled and disabled in the Bios but basically none when TDX was enabled in the Bios but disabled in the VM. Further investigations into what causes these speed-ups were necessary. As stated in \ref{sec:tdxBuildingBlocks} there were essentially three functionalities that could be tested independently from TDX:
\begin{itemize}
    \item Total Memory Encryption 
    \item Intel VT, virtualization features
    \item Intel SGX
\end{itemize}

They were tested in the order outlined here.
For Total Memory Encryption, TDX and SGX were disabled, while Intel VT remained turned on. Stream \cite{Stream} and "PerformanceTest" \url{PerformanceTest} were used for benchmarking. Stream is the de facto standard for memory speed testing and PerformanceTest is a commonly used proprietary tool used to compare memory access. Using PerformanceTest there were small but measurable differences in latency of about 5\% and 2.5\% for reading speed. The Writing speed remained within 1\% of each other.
\begin{table}
\centering
\resizebox{\textwidth}{!}{%
\begin{tabular}{ m{0.37\textwidth} m{0.25\textwidth} m{0.25\textwidth} m{0.13\textwidth}}
\toprule
Memory Benchmark Type & Memory encryption activated & Memory encryption deactivated & Difference \\
\midrule
Memory Allocation MB/s & 32445.11 & 33258.94 & 2.5\% \\
\midrule
Memory Read Cached MB/s & 27683.36 & 28148.14 & 1.6\% \\
\midrule
Memory Read Uncached MB/s & 10879.98 & 11123.77 & 2.2\% \\
\midrule
Memory Write Medium MB/s & 9592.82 & 9549.92 & -0.4\% \\
\midrule
Memory Write Threaded MB/s & 400550.9 & 411297.5 & 2.6\% \\
\midrule
Memory Latency ns & 64.3 & 61.4 & -4.5\% \\
\bottomrule
\end{tabular}
}
\caption{Results of Performancetests memory benchmark}
\label{tab:MemoryAccessSpeed}
\end{table}

With Stream, everything was tested disabled, just Memory Encryption enabled, and TDX completely enabled, and all three were within two percent of one another, with the encrypted memory being slightly faster than the other two. These results are most likely due to background noise of the system. \ref{fig:STREAM full results}

\begin{table}[]
    \centering
    \resizebox{\textwidth}{!}{%
    \begin{tabular}{m{0.2\textwidth} m{0.2\textwidth} m{0.2\textwidth} m{0.2\textwidth} m{0.2\textwidth} m{0.2\textwidth}}
    \hline
        Function & Best Rate MB/s Encrypted & Best Rate MB/s TDX enabled & Best Rate MB/s unencrypted & Best Rate Legacy VM MB/s & Best Rate TD MB/s \\ \hline
        Copy: & 292927.7 & 296714.7 & 292686.3 & 228281.2 & 209348.5979 \\ \hline
        Scale: & 291833.5 & 296603.3 & 293131.2 & 225828 & 206950.9938 \\ \hline
        Add: & 324643.9 & 318742.3 & 317304.6 & 250686.5 & 221401.8875 \\ \hline
        Triad: & 315635.1 & 329435.5 & 314132 & 232469.4 & 225711.5192 \\ \hline
        Normalized Geometric Mean & 1.418 & 1.437 & 1.410 & 1.085 & 1.000 \\ \hline
    \end{tabular}
    }
    \caption{Comparison of Memory Access speed using STREAM for various Non-VM settings as well as legacy VMs and TD}
\end{table}

Intel

\subsection{Security Attestation}

Attestation was not implemented or usable on the Intel Developer Cloud. Verifying usage of TDX was thus relying on the Linux Kernel itself as well as Bios settings made by the user via an SSH connection. Without attestation the security benefits were theoretically not given. This was accepted as having bare-metal access meant being able to verify TDX activation directly.


\subsection{Setting up the benchmarks}
\label{sec:SecondContent:SecondSection}

The benchmark will be run in Python, as it is the most commonly used programming language in natural language processing. The benchmarks will predominantly be an inference benchmark using distillBERT and roBERTa-base, the code can be found in the appendix \ref{app:BenchmarkCode}. All benchmarks will be run using the same image on their own QEMU KVM with 32 Cores and 64GB of RAM on a machine containing 2 Intel Xeon 8490H with each 60 cores @1.90 Ghz, except if stated otherwise. The setup and boilerplate code can be found in the appendix aswell \ref{app:Setup}. Performance is measured with Pyperf and each test is run 30 times with the median and the standard deviation being shown.
For the inference Benchmark a short sentence with 16 tokens and a long sentence with 146 tokens were used. The long sentence was supposed to have 128 tokens but due to an oversight the sentence was a bit longer than initially thought. Additional more fine-grained benchmarks were done later on.



\section{Application Possibilities}

\todo{Wirklich sicher?}
\todo{Was ist der Nutzervorteil?}
\todo{Wie komplex ist die Nutzung?}
%% ---------------------
%% | / Example content |
%% ---------------------



\section{Outlook}

This thesis focused on simple TDX implementations. With TDX having been just released, most of those implementations are still in their infancy and problems are to be expected. Hopefully in the future some of those, that have been highlighted in this thesis and those that have not been will be fixed. Nonetheless there are still additional things that can be looked at. The implementation of TDX in Ubuntu 22 appears to cause instability issues and also performance defiencies compared to the implementation in Ubuntu 23. Having a deeper look at the differences between those two could grant significant improvements. Additionally, comparisons between the different hardware Confidential Computing hardware providers are necessary to being able to make the decision for or against Confidential Computing. Additionally looking into the implementation of PyTorch to figure out why its copy function experiences such significant slowdown is advised. 

For the future Intel has planned on a cooperation with Nvidia to bring Confidential Computing and AI closer together \url{https://community.intel.com/t5/Blogs/Products-and-Solutions/Security/Intel-Nvidia-Collaborate-to-Deliver-Confidential-AI-Solutions/post/1500066}. With the release of Intel attestation for Nvidia hardware to be planned in the first half of 2024, confidential computing on GPU hardware could be a possibility. Intel and Nvidia have yet to release architecture specifications for their collaboration but their security promises and assumptions will have to be tested then. If implementation could be simplified for the average user this could be a great step into the right direction for data security. The dangers of dedicated GPUs and the possible performance losses with establishing a secure communication between CPU and GPU are a challenge that will have to be solved first. Silberstein et al. highlighted the dangers of insecure communication and other problems with dedicated GPUs in their 2020 paper \cite{Silberstein_GPUAttack}.

Additionally this thesis was limited in regard to the amount of performancetesting that was feasible. More I/O heavy workloads and even some GPU-bound workloads could be tested. Implementing ways to calculate on a non-trusted GPU, while maintaining data confidentiality is still an ongoing line of research, which could be even more interesting in the future, but it still has security and performance problems, compare \cite{Monique Ogburn} and \cite{Baiyu Li} for more information on this.


\todo{Integration with GPUs, FPGAs, TPUs,...}

%% LaTeX2e class for student theses
%% sections/conclusion.tex
%% 
%% Karlsruhe Institute of Technology
%% Institute for Program Structures and Data Organization
%% Chair for Software Design and Quality (SDQ)
%%
%% Dr.-Ing. Erik Burger
%% burger@kit.edu
%%
%% Version 1.4, 2023-06-19

\chapter{Conclusion}
\label{ch:Conclusion}

The general performance impacts of TDX are so small that they can be mostly ignored in day-to-day applications. Most users will not notice any differences, especially if memory encryption is used anyway. Comparing hosting on your own hardware, hosting in the cloud with TDX and hosting in the cloud without TDX, only about 10\% calculation time differences are to be expected. Bigger improvements could be made if a platform that does not need an additional VM layer can be made available, this could improve calculation speeds up to 30\%. There are however certain scenarios that could lead to a much higher overhead. As shown in \cref{performance} some instructions or methods can experience significant slowdowns when using TDX. It is unclear what leads to these slowdowns so the developer needs to check for themselves if they experience some unexplainable slow execution speeds.

These small performance decreases can potentially buy some great security guarantees, especially against software based attacks from a malicious Host or similar. TDX attestation, if implemented properly, seems to be found in its security assumptions. The threat model considers anyone with hardware access as untrusted, this includes the cloud provider. As discussed previously TDX can not protect against all hardware attacks, meaning that anyone with hardware access can extract information from a TD if they know how to implement certain kinds of attacks, discussed in \cref{Security Analysis}. Similarly someone with hardware access can also extract information on self-hosted applications. The assumption that the cloud provider can be completely removed from the list of trusted parties, as claimed by Intel, is thus at best an exaggeration and at worst just false. Looking further into one implementation of Intel TDX these issues become even more pronounced, a lot of trust has to be put into Microsoft proprietary code regarding the attestation. It was not possible for me to create a quote that was completely independent of any Microsoft code. As it stands currently it is not recommended to rely upon TDX, to remove the cloud provider from the list of trusted parties.


\myparagraph{Outlook}

This thesis focused on simple TDX implementations. With TDX having just been released, most of those implementations are still in their infancy and problems are to be expected. Hopefully in the future some of those, that have been highlighted in this thesis and those that have not been will be fixed. Nonetheless there are still additional things that can be looked at. The implementation of TDX in Ubuntu 22 appears to cause instability issues and also performance defiencies compared to the implementation in Ubuntu 23. Having a deeper look at the differences between those two could grant significant improvements. Additionally, comparisons between the different hardware Confidential Computing hardware providers are necessary to being able to make the decision for or against Confidential Computing. Additionally looking into the implementation of PyTorch to figure out why its copy function experiences such significant slowdown is advised. 

For the future Intel has planned a cooperation with Nvidia to bring Confidential Computing and AI closer together \url{https://community.intel.com/t5/Blogs/Products-and-Solutions/Security/Intel-Nvidia-Collaborate-to-Deliver-Confidential-AI-Solutions/post/1500066}. With the release of Intel attestation for Nvidia hardware to be planned in the first half of 2024, confidential computing on GPU hardware could be a possibility. Intel and Nvidia have yet to release architecture specifications for their collaboration but their security promises and assumptions will have to be tested then. If implementation could be simplified for the average user this could be a great step into the right direction for data security. The dangers of dedicated GPUs and the possible performance losses with establishing a secure communication between CPU and GPU are a challenge that will have to be solved first. Silberstein et al. highlighted the dangers of insecure communication and other problems with dedicated GPUs in their 2020 paper \cite{zhu_enabling_2020}. All of these improvements and collaborations are only nice to have as long as the issues outlined in this thesis are not remedied.

Additionally this thesis was limited in regard to the amount of performancetesting that was feasible. More I/O heavy workloads and even some GPU-bound workloads could be tested. Implementing ways to calculate on a non-trusted GPU, while maintaining data confidentiality is still an ongoing line of research, which could be even more interesting in the future, but it still has security and performance problems, compare \cite{ogburn_homomorphic_2013} and \cite{li_security_2020} for more information on this.


\input{sections/relatedWork.tex}


%% --------------------
%% |   Bibliography   |
%% --------------------

%% Add entry to the table of contents for the bibliography
\printbibliography[heading=bibintoc]

%% ----------------
%% |   Appendix   |
%% ----------------
\appendix
%% LaTeX2e class for student theses
%% sections/apendix.tex
%% 
%% Karlsruhe Institute of Technology
%% Institute for Program Structures and Data Organization
%% Chair for Software Design and Quality (SDQ)
%%
%% Dr.-Ing. Erik Burger
%% burger@kit.edu
%%
%% Version 1.4, 2023-06-19

\iflanguage{english}
{\chapter{Appendix}}    % english style
{\chapter{Anhang}}      % german style
\label{chap:appendix}


%% -------------------
%% | Example content |
%% -------------------

\section{Security}

\paragraph{Microsoft Azure Attestation JWT verification}
\label{jwt}
\begin{lstlisting}[language=json]
    {
  "attester_tcb_status": "UpToDate",
  "dbgstat": "disabled",
  "eat_profile": "https://aka.ms/maa-eat-profile-tdxvm",
  "exp": 1705689946,
  "iat": 1705661146,
  "intuse": "generic",
  "iss": "https://sharedeus2e.eus2e.attest.azure.net",
  "jti": "01bd3ce9833084751244fbde515a9c410f2af383ee483734b410d8c031120018",
  "nbf": 1705661146,
  "tdx_mrconfigid": "000000000000000000000000000000000000000000000000000000000000000000000000000000000000000000000000",
  "tdx_mrowner": "000000000000000000000000000000000000000000000000000000000000000000000000000000000000000000000000",
  "tdx_mrownerconfig": "000000000000000000000000000000000000000000000000000000000000000000000000000000000000000000000000",
  "tdx_mrseam": "360304d34a16aace0a18e09ad2d07d2b9fd3c174378e5bf108388079827f89ff62acc5f8c473dd40706324834e202946",
  "tdx_mrsignerseam": "000000000000000000000000000000000000000000000000000000000000000000000000000000000000000000000000",
  "tdx_mrtd": "024a32b070383331181619fa387cb4d55d1e38879f989933055ccad5bc2db795d1737b66205949d15469dc8c1ba7ab7b",
  "tdx_report_data": "c90f98ba8ab80c7b442b6b8eb30af54e0508077b11adb525af6dfbcc8714e52a0000000000000000000000000000000000000000000000000000000000000000",
  "tdx_rtmr0": "000000000000000000000000000000000000000000000000000000000000000000000000000000000000000000000000",
  "tdx_rtmr1": "000000000000000000000000000000000000000000000000000000000000000000000000000000000000000000000000",
  "tdx_rtmr2": "000000000000000000000000000000000000000000000000000000000000000000000000000000000000000000000000",
  "tdx_rtmr3": "000000000000000000000000000000000000000000000000000000000000000000000000000000000000000000000000",
  "tdx_seam_attributes": "0000000000000000",
  "tdx_seamsvn": 258,
  "tdx_td_attributes": "0000000000000000",
  "tdx_td_attributes_debug": false,
  "tdx_td_attributes_key_locker": false,
  "tdx_td_attributes_perfmon": false,
  "tdx_td_attributes_protection_keys": false,
  "tdx_td_attributes_septve_disable": false,
  "tdx_tee_tcb_svn": "02010600000000000000000000000000",
  "tdx_xfam": "e718060000000000",
  "x-ms-attestation-type": "tdxvm",
  "x-ms-compliance-status": "azure-compliant-cvm",
  "x-ms-policy-hash": "9NY0VnTQ-IiBriBplVUpFbczcDaEBUwsiFYAzHu_gco",
  "x-ms-runtime": {
    "keys": [
      {
        "e": "AQAB",
        "key_ops": [
          "sign"
        ],
        "kid": "HCLAkPub",
        "kty": "RSA",
        "n": "p4ig6gABSEmW9oijgg37ZDj7ZcaxnyN2h9KTHLAUYJnoP-42Hi209p1RdiL3HSWdRzzSDuXKjQw2kzlmcX1XfBq9-3-L-CVoyCvpiin-6k9L_Rbu0upEiBnIw2IzJ5N6EZx7zfX8vdh76MMnT-_U2Pd2psoiUufzNvs2-5d0QB5NyTuCnXbWD_yWb1OQWZAPioXEIyR13ic-FxPt5UiGG6kBjGp41rqlEQ0B_tsfn3e19_lTx76wVfEkEOw5Yq60rTyNuxiOnR59reaxXui0qJTOkoiDOJ-tHLffY_jTT-8EUFtWS1mcPoye2uKRhAI99xjTWg88Ft4sIUv8stOTvw"
      },
      {
        "e": "AQAB",
        "key_ops": [
          "encrypt"
        ],
        "kid": "HCLEkPub",
        "kty": "RSA",
        "n": "4galYAAA8uQxZCGg7qvvY0hcck4UY-EPiqWyxkHH0rMyQJh2nWuriu5wrh4JPPy6HvtX5M-mr5ecHPYf4S0qTOcqWZX01Uj8Ky1sbvA6g1GELe0Jdqf9piQDmS90dGwR6PItHMObnejzBocweHxkuSm7pGSuOAniX6KDid1hsxED5gYLxg2wHAXFICz9UZOTQ6FZmezVz0Krpr2AF4fKawOIaQSwNl7GIjH09x7rFRP2H1zOW6sJYAIAIT7F7vfncB3ZpPHUrU9U-qSQC2LZBRtjuBaKG-kGU6jAOkMC8aXu_pb99vgrrJpn2Nf8VodePJ7gzwladZjCnflyUWViYw"
      }
    ],
    "user-data": "00000000000000000000000000000000000000000000000000000000000000000000000000000000000000000000000000000000000000000000000000000000",
    "vm-configuration": {
      "console-enabled": true,
      "root-cert-thumbprint": "6nZZnYaJc4KqUZ_yvA-mucFdYNouvlPnITnNMXsHl-0",
      "secure-boot": true,
      "tpm-enabled": true,
      "tpm-persisted": true,
      "vmUniqueId": "C540205B-C5A0-4380-A0F9-7ECA9B91113A"
    }
  },
  "x-ms-ver": "1.0"
}
\end{lstlisting}


\end{document}
